\documentclass{article}
\usepackage[T1]{fontenc}
\usepackage{beramono}% monospaced font with bold variant

\usepackage{listings}
\lstdefinelanguage{VHDL}{
	morekeywords=[1]{
		library,use,all,entity,is,port,in,out,end,architecture,of,
		begin,and,or,not,xor,nor,xnor,buffer,not,if,then,else,elsif,
		when,others,signal,process,variable,constant
	},
	morekeywords=[2]{
		std_logic,std_logic_vector,std_ulogic,std_ulogic_vector
	},
	morecomment=[l]--
}

\usepackage{xcolor}
\colorlet{keyword}{blue!100!black!80}
\colorlet{comment}{green!90!black!90}
\lstdefinestyle{vhdl}{
	language=VHDL,
	basicstyle=\small\ttfamily,
	keywordstyle=[1]\color{blue}\bfseries,
	keywordstyle=[2]\color{red}\bfseries,
	commentstyle=\color{green!50!black},
	tabsize=4,
	showstringspaces=false,
	breaklines=true,
	numbers=left,
	numberstyle=\tiny\color{gray},
	stepnumber=1,
	numbersep=10pt,
	frame=lines
}

\begin{document}
\section{Example VHDL code}
Here's an example of how to include VHDL code in a \LaTeX document using the \texttt{listings} package:
	
\begin{lstlisting}[style=vhdl,caption={Example VHDL code}]

library ieee;
use ieee.std_logic_1164.all;

entity example is
port (
clk : in std_logic;
rst : in std_logic;
data_in : in std_logic_vector(7 downto 0);
data_out : out std_logic_vector(7 downto 0)
);
end entity;

architecture rtl of example is
begin
-- insert your VHDL code here
end architecture;
\end{lstlisting}
	
\end{document}