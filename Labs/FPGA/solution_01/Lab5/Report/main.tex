% Exercise Template
%A LaTeX template for typesetting exercise in Persian (with cover page).
%By: Reza Adinepour

\documentclass[12pt]{exam}

\usepackage{setspace}
\usepackage{listings}
\usepackage{graphicx,subfigure,wrapfig}
\usepackage{multirow}
\usepackage{matlab-prettifier}
\usepackage{amsmath}
\usepackage{xcolor}
\usepackage{multicol}
\usepackage[T1]{fontenc}
\usepackage{beramono}% monospaced font with bold variant


\usepackage[margin=20mm]{geometry}
\usepackage{xepersian}
\settextfont{XB Niloofar}

\newcommand{\class}{آزمایشگاه FPGA}
\newcommand{\term}{نیم‌سال دوم ۰۱-۰۲}
\newcommand{\college}{دانشکده مهندسی برق}
\newcommand{\prof}{استاد: دکتر خرقانیان}

\singlespacing
\parindent 0ex

\begin{document}


% -------------------------------------------------------
%  Thesis Information
% -------------------------------------------------------

\newcommand{\ThesisType}
{سمینار}  % پایان‌نامه / رساله
\newcommand{\ThesisDegree}
{کارشناسی ارشد گرایش معماری کامپیوتر}  % کارشناسی / کارشناسی ارشد / دکتری
\newcommand{\ThesisMajor}
{مهندسی برق}  % مهندسی کامپیوتر
\newcommand{\ThesisTitle}
{ساعت دیجیتال}
\newcommand{\ThesisAuthor}
{رضا آدینه پور-9814303\\علی‌رضا قربانی-9823263}
\newcommand{\ThesisSupervisor}
{جناب آقای دکتر رضا خرقانیان}
\newcommand{\ThesisDate}
{خرداد 1402}
\newcommand{\ThesisDepartment}
{دانشکده مهندسی برق}
\newcommand{\ThesisUniversity}
{دانشگاه صنعتی شاهرود}

% -------------------------------------------------------
%  English Information
% -------------------------------------------------------

\newcommand{\EnglishThesisTitle}{A Standard Template for Course Exercise}


\pagestyle{empty}
\include{cover-page}

% These commands set up the running header on the top of the exam pages
\pagestyle{head}
\firstpageheader{}{}{}
\runningheader{صفحه \thepage\ از \numpages}{}{\class}
\runningheadrule

\vspace{0pt}

\lstdefinelanguage{VHDL}{
	morekeywords=[1]{
		library,use,all,entity,is,port,in,out,end,architecture,of,
		begin,and,or,not,xor,nor,xnor,buffer,not,if,then,else,elsif,
		when,others,signal,process,variable,constant
	},
	morekeywords=[2]{
		std_logic,std_logic_vector,std_ulogic,std_ulogic_vector
	},
	morecomment=[l]--
}


\colorlet{keyword}{blue!100!black!80}
\colorlet{comment}{green!90!black!90}
\lstdefinestyle{vhdl}{
	language=VHDL,
	basicstyle=\small\ttfamily,
	keywordstyle=[1]\color{blue}\bfseries,
	keywordstyle=[2]\color{red}\bfseries,
	commentstyle=\color{green!50!black},
	tabsize=4,
	showstringspaces=false,
	breaklines=true,
	numbers=left,
	numberstyle=\tiny\color{gray},
	stepnumber=1,
	numbersep=10pt,
	frame=lines
}




\begin{questions}
\pointpoints{نمره}{نمره}

\question
با استفاده از کد \texttt{VHDL} یک ساعت دیجیتال با تکنیک مالتی‌پلکس زمانی پیاده سازی کنید. \\ \\
در این گزارش کد آزمایش قبل که صرفا ثانیه و دقیقه را نشان میداد تکمیل کرده و یک ساعت دیجیتال 24 ساعته با تکنیک مالتی‌پلکس زمانی نوشته ایم. \\ \\
• کد نوشته شده به‌صورت زیر است: 


\begin{latin}
\begin{lstlisting}[style=vhdl,caption={Example VHDL code}]
	
library IEEE;
use IEEE.STD_LOGIC_1164.ALL;
use IEEE.STD_LOGIC_ARITH.ALL;
use IEEE.STD_LOGIC_UNSIGNED.ALL;

entity main is
Port (  b 	: in std_logic;
		clk : in std_logic;
		Q	: out std_logic_vector(5 downto 0);
		dpp : out std_logic;
		Z 	: out std_logic_vector(0 to 6);
		rest: in std_logic );
end main;

architecture Behavioral of pr is
	signal counter1 : std_logic_vector(3 downto 0) := "0000";
	signal counter2 : std_logic_vector(3 downto 0) := "0000";
	signal counter3 : std_logic_vector(3 downto 0) := "0000";
	signal counter4 : std_logic_vector(3 downto 0) := "0000";
	signal counter5 : std_logic_vector(3 downto 0) := "0000";
	signal counter6 : std_logic_vector(3 downto 0) := "0000";
	signal clk_1s : std_logic := '0';
	function Bcd_7seg( s : std_logic_vector(3 downto 0)) return std_logic_vector is                           
	variable y : std_logic_vector(6 downto 0);
begin
	case (s) is
		when "0000" =>
			y:= "1111110";
		when "0001" =>
			y:= "0110000";
		when "0010" =>
			y:= "1101101";
		when "0011" =>
			y:= "1111001";
		when "0100" =>
			y:= "0110011";
		when "0101" =>
			y:= "1011011";
		when "0110" =>
			y:= "1011111";
		when "0111" =>
			y:= "1110000";
		when "1000" =>
			y:= "1111111";
		when "1001" =>
			y:= "1111011";
		when others =>
			y:= "0000000";
	end case;
	return y;
end Bcd_7seg;

begin
	dpp <= b;
	process (clk)
	variable counter0 : integer range 0 to 1000:= 0;
	begin
		if (clk'event and clk = '1') then
			counter0 := counter0 + 1;
			if (counter0 < 500) then
				clk_1s <= '0';
			elsif (counter0 >= 500) then
				clk_1s <= '1';
			end if;
		end if;
	end process;
	
	process (rest, clk_1s)
	begin
		if(rest = '0') then
			counter1 <= "0000";
			counter2 <= "0000";
			counter3 <= "0000";
			counter4 <= "0000";
			counter5 <= "0000";
			counter6 <= "0000";
		else
			if (clk_1s'event and clk_1s='1')then
				counter1 <= counter1+1;
				if (counter1 = "1001") then
					counter1 <= "0000";
					counter2 <= counter2+1;
					if (counter2 = "0101") then
						counter2 <= "0000";
						counter3 <= counter3+1;
						if (counter3 = "1001") then
							counter3 <= "0000";
							counter4 <= counter4+1;
							if (counter4 = "0101") then
								counter4 <= "0000";
								counter5 <= counter5+1;
								if (counter5 = "1001") then
									counter5 <= "0000";
									counter6 <= counter6+1;
									if (counter6 = "0010") then
										counter6 <= "0000";
										counter5 <= "0000";
										counter4 <= "0000";
										counter3 <= "0000";
										counter2 <= "0000";
										counter1 <= "0000";				 	
									end if;	
								end if;
							end if;
						end if;
					end if; 
				end if;
			end if;
		end if;
	end process;

	process(clk)
	variable C: integer range 0 to 5 := 0;
	begin
		if (clk'event and clk='1') then
			C:=C+1;
			if (C=1) then 
				Q <= "000001";
				Z <= Bcd_7seg(counter1);
			elsif (C = 2) then
				Q <= "000010";
				Z <= Bcd_7seg(counter2);
			elsif(c = 3) then
				Q <= "000100";
				Z <= Bcd_7seg(counter3);
			elsif(c = 4) then
				Q <= "001000";
				Z <= Bcd_7seg(counter4);
			elsif(c = 5) then
				Q <= "010000";
				Z <= Bcd_7seg(counter5);
			elsif(c = 6) then
				Q <= "100000";
				Z <= Bcd_7seg(counter6);				
			end if;
		end if;
	end process;
end Behavioral;
\end{lstlisting}
\end{latin}

\end{questions}
\end{document}